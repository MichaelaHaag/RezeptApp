\chapter{Programming Principles}
Programming principles (Programmierprinzipien) sind Grundsätze, die beim Entwurf, der Entwicklung und dem Testen von Software angewendet werden können. Sie sind eine Verallgemeinerung wiederkehrender Erkenntnisse in der Softwareentwicklung und liefern Entwicklern Richtlinien für einen bestimmten Programmierstil.
Im Allgemeinen zielen sie darauf ab, die Qualität und Robustheit von Software zu verbessern. Im folgenden Abschnitt werden drei Programming principles vorgestellt und deren Anwendung in unserem Projekt analysiert. 
\section{SOLID}
Die SOLID-Prinzipien sind ein Konzept für objektorientierte Programmierung, das fünf Grundsätze für die Entwicklung von hochwertigem und wartbarem Code definiert. Indem man diese Prinzipien befolgt, kann man sicherstellen, dass der Code besser strukturiert und leichter zu erweitern ist, und dass er weniger fehleranfällig ist und besser gewartet werden kann. Jeder Buchstabe in SOLID steht für einen dieser Grundsätze:
\subsection{Single Responsibility Principle (SRP)}
Das Single Responsibility Principle (SRP) besagt, dass ein einzelnes Objekt oder eine Klasse nur für eine einzige Aufgabe oder Verantwortlichkeit zuständig sein  sollte.
In unserer Anwendung ist beispielsweise eine Klasse, die dieses Prinzip strikt einhält: FunktionenZufallsGenerator. Diese Klasse hat lediglich die Aufgabe, aus einer Liste von Rezepten ein "zufälliges" Rezept auszuwählen. 
Ein Negativbeispiel für eine Klasse, die das Single Responsibility Principle nicht einhält, ist unser EntityManager. Unser EntityManager wird verwendet, um eine Verbindung zwischen den Objekten der Anwendung und der zugrunde liegenden Datenbank herzustellen. Mit dem EntityManager wollten wir die Verwaltung von Objekten und deren Zuständen vereinfachen. Das bedeutet allerdings, dass der EntityManager mehrere Aufgaben (Verantwortungen) hat und somit das Prinzip verletzt. 
Andere Positivbeispiele für die Einhaltung des Single Responsibility Principle sind beispielsweise die Klassen FunktionenRezeptBearbeiten, FunktionenNeuesRezept, ButtonRenderer und FunktionenListenÜbersicht. Die Klasse FunktionenListenÜbersicht ist beispielsweise nur dafür verantwortlich alle Rezepte zu einer angeklickten Kategorie zurückzugeben. 