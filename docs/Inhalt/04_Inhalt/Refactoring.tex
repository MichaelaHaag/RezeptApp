\chapter{Refactoring}
Refaktorisierung (auch als Refactoring bezeichnet) ist ein Prozess in der Softwareentwicklung, bei dem der Code einer Anwendung geändert wird, ohne das Verhalten der Anwendung selbst zu ändern. Das Ziel von Refaktorisierung ist es, den Code zu verbessern, indem er einfacher, verständlicher und wartbarer gemacht wird.

Während der Entwicklung von Software kann es vorkommen, dass der Code mit der Zeit unübersichtlich und komplex wird. Dies kann dazu führen, dass Änderungen oder Erweiterungen an der Software schwierig und fehleranfällig sind. Durch Refaktorisierung kann der Code so umstrukturiert werden, dass er einfacher zu verstehen und zu warten ist.