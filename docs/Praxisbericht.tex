% ------------------------------------------------------------
% LaTeX Template für die DHBW zum Schnellstart!
% Original: https://github.wdf.sap.corp/vtgermany/LaTeX-Template-DHBW
% ------------------------------------------------------------
% ---- Präambel mit Angaben zum Dokument
\input{Inhalt/00_Latex/praeambel}

% ---- Elektronische Version oder Gedruckte Version?
% ---- Unterschied: Die elektronische Version enthält keinen Platzhalter für die Unterschrift
\usepackage{ifthen}
\newboolean{e-Abgabe}
\setboolean{e-Abgabe}{true}    % false=gedruckte Fassung

% ---- Persönlichen Daten:
\newcommand{\titel}{Programmentwurf}
\newcommand{\titelheader}{Programmentwurf}
\newcommand{\arbeit}{Advanced Software Engineering (T3INF3001)}
\newcommand{\studiengang}{Informatik}
\newcommand{\studienjahr}{2015}
\newcommand{\autor}{Katharina Braun (7032223)}
\newcommand{\autorzwei}{Michaela Haag (3098671)}
\newcommand{\autorReverse}{Braun, Katharina}
\newcommand{\autorzweiReverse}{Haag, Michaela}
\newcommand{\verfassungsort}{Karlsruhe}
\newcommand{\kurs}{TINF20B1}
\newcommand{\abgabe}{30. April 2023}
\newcommand{\bearbeitungszeitraum}{03.10.2022 - 30.04.2023}
\newcommand{\betreuerDhbw}{Daniel Lindner}

\input{Inhalt/00_Latex/kopfundFusszeile}

% ---- Hilfreiches
\newcommand{\zB}{z.\,B. }   % "z.B." mit kleinem Leeraum dazwischen (ohne wäre nicht korrekt)
\newcommand{\dash}{d.\,h. }

\newcommand{\code}[1]{\texttt{#1}} % Ist einfacher zu schreiben als ständig \texttt und erlaubt
                                   % Änderungen im Nachhinein, wenn man z.B. Inline-Code anders stylen möchte.

% ---- Silbentrennung (falls LaTeX defaults falsch / nicht gewünscht sind)
\hyphenation{HANA}         % anstatt HA-NA
\hyphenation{Graph-Script} % anstatt GraphS-cript

% ---- Beginn des Dokuments
\begin{document}
\setlength{\parindent}{0pt}              % Keine Paragraphen Einrückung.
                                         % Dafür haben wir den Abstand zwischen den Paragraphen.
\setcounter{secnumdepth}{2}              % Nummerierungstiefe fürs Inhaltsverzeichnis
\setcounter{tocdepth}{1}                 % Tiefe des Inhaltsverzeichnisses. Ggf. so anpassen,
                                         % dass das Verzeichnis auf eine Seite passt.
\sffamily                                % Serifenlose Schrift verwenden.

% ---- Vorspann
% ------ Titelseite
\singlespacing
\include{Inhalt/01_Standard/titelseite}  % Titelseite
\newcounter{savepage}
\pagenumbering{Roman}                    % Römische Seitenzahlen
\onehalfspacing

% ------ Inhaltsverzeichnis
\singlespacing
\tableofcontents

% ------ Verzeichnisse
\renewcommand*{\chapterpagestyle}{plain}
\pagestyle{plain}
% \include{Inhalt/03_Verzeichnisse/formelgroessen}
% \include{Inhalt/03_Verzeichnisse/abkuerzungen}
\listoffigures                          % Erzeugen des Abbildungsverzeichnisses 
% \listoftables                           % Erzeugen des Tabellenverzeichnisses
% \renewcommand{\lstlistlistingname}{Quellcodeverzeichnis}
% \lstlistoflistings                      % Erzeugen des Listenverzeichnisses
\setcounter{savepage}{\value{page}}


% ---- Inhalt der Arbeit
\cleardoublepage
\pagenumbering{arabic}                  % Arabische Seitenzahlen für den Hauptteil
\setlength{\parskip}{0.5\baselineskip}  % Abstand zwischen Absätzen
\rmfamily
\renewcommand*{\chapterpagestyle}{scrheadings}
\pagestyle{scrheadings}
\onehalfspacing
%\chapter{Einleitung}
Wir haben uns entschieden, für das Advanced Software-Engineering Projekt,
eine von uns entwickelte Rezeptverwaltung zu verwenden. Diese Anwendung
soll später auch als mobile App umgesetzt werden, weshalb dafür eine
Codeverbesserung sehr sinnvoll ist.
\section{Funktionalität}
Die Anwendung wurde entwickelt, um Personen die Planung Ihrer Mahlzeiten zu vereinfachen.
In der Anwendung ist es möglich, alle Rezepte zentral an einer Stelle zu verwalten, anstelle von
mehreren verteilen Rezeptbüchern. Beim Anlegen neuer Rezepte ist es neben der Angabe des
Namen, der Zutaten, der Beschreibung, des Schwierigkeitsgrads und eines Bildes auch möglich, das
Rezept vordefinierten Kategorien (z. B. Grundzutaten) zuzuordnen.
Auf der Startseite der App gibt es dann die Möglichkeit, sich eine Liste aller verfügbaren Rezepte
anzuzeigen oder nur die Rezepte einzelner Kategorien. Wird in der Rezeptliste ein Rezept
ausgewählt, so bekommt der Anwender eine Detailansicht des Rezepts.
Des Weiteren hat die Anwendung noch die Funktion, dass der Anwender sich ein zufälliges Rezept
generieren kann. Möchte der Anwender das Rezept kochen, dann kann er sich die Details des
Rezepts anzeigen lassen. Wenn der Anwender mit der Auswahl unzufrieden ist, so gibt es die
Möglichkeit, ein neues zufälliges Rezept generieren zu lassen.
\section{Kundennutzen}
Der Kundennutzen liegt darin, dass Anwender alle Ihre Lieblingsrezepte an einem Ort sammeln und
nach Ihren Wünschen organisieren können. Dadurch verhindern wir langes recherchieren nach
einem online Rezept oder das Suchen nach Omas Rezepten auf einem Zettel.
Nutzer können einfach eigene Rezepte anlegen und ein Foto eines Rezepts hochladen, z. B. aus
einem Buch oder einer Zeitschrift.
Außerdem möchten wir die Vielfalt beim Essen vergrößern, indem der Nutzer ein Zufall Rezept aus
einem großen Pool von Rezepten vorgeschlagen bekommt oder indem er direkt nach Kategorien
filtert.
Mit unserer Anwendung erleichtern wir somit den Anwendern die Entscheidung, welches Gericht
jeden Tag gekocht werden soll.
\section{Technologie}
Die Anwendung ist aktuell in Java entwickelt und als Architektur wurde das MVC-Konzept verwendet
(Trennung von Daten, Benutzeroberfläche und Logik).
Für die Datenhaltung haben wir uns entschieden, alle Daten in CSV-Dateien zu speichern. 

Der Code dieser Anwendung ist in dem folgenden GitHub Repository gespeichert: 
\href{https://github.com/MichaelaHaag/RezeptApp/}{https://github.com/MichaelaHaag/RezeptApp/}

% ---- Literaturverzeichnis
\cleardoublepage
\renewcommand*{\chapterpagestyle}{plain}
\pagestyle{plain}
\pagenumbering{Roman}                   % Römische Seitenzahlen
\setcounter{page}{\numexpr\value{savepage}+1}
\printbibliography[title=Literaturverzeichnis]

% ---- Anhang
\appendix
%\clearpage
%\pagenumbering{Roman}  % römische Seitenzahlen für Anhang

\newpage
\end{document}
